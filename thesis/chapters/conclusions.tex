This project investigated the application of federated learning for anomaly-based DDoS detection in heterogeneous network environments, with a particular emphasis on evaluating optimization strategies for improving model convergence under non-IID data distributions. The study built a simulation framework using the Flower federated learning library, leveraging its partitioning mechanisms to create varying levels of heterogeneity across multiple clients. By systematically experimenting with algorithms such as FedAvg, FedProx, FedAdam, and FedYogi, this work sought to understand the trade-offs between client-side and server-side optimization strategies when deployed in a distributed anomaly detection context.

The research was motivated by the increasing scale and sophistication of DDoS attacks, which require distributed and privacy-preserving solutions for detection. Traditional centralized approaches face significant challenges related to communication overhead, privacy risks, and scalability. Federated learning, by design, mitigates many of these challenges by keeping data localized while allowing models to benefit from shared training. However, the heterogeneity of client data and resource constraints remain core bottlenecks for its adoption. Addressing these issues through improved optimization methods is therefore critical.

The results obtained in this work demonstrated autoencoders are well-suited for anomaly detection in network security, as they can effectively model normal traffic patterns and identify deviations indicative of attacks. Results also provide several important insights into the performance of different federated optimization strategies. FedAvg served as the foundational baseline for comparison. While it demonstrated acceptable convergence in relatively homogeneous settings, its performance degraded significantly under more heterogeneous client distributions (Dirichlet $\alpha=0.3$), particularly when scaling beyond 20 clients. This reinforced prior findings that FedAvg alone is insufficient for highly non-IID environments. FedProx consistently outperformed FedAvg in settings with high heterogeneity. By introducing a proximal term that reduces client-side drift, FedProx stabilized local updates and led to improved model convergence across most experimental setups. Notably, FedProx achieved robust gains across both 20- and 50-client simulations, demonstrating its capacity to handle diverse data distributions effectively. These results make FedProx a reliable approach across most federated scenarios. By contrast, FedAdam, a server-side adaptive optimization method inspired by Adam, only demonstrated clear improvements over FedAvg in the largest-scale setup (50 clients). In smaller-scale or less heterogeneous scenarios, its sensitivity to noisy updates often led to performance degradation relative to FedAvg. Therefore, this result suggests that FedAdam and other adaptive optimizers are often suitable for highly scaled systems. FedProx and FedAdam were also combined to be tested in 50-client setups, and they indeed achieved performance gains, with some results approaching the centralized training results even under strong non-IID conditions. Moreover, tests conducted on CIC-DDoS2019 consistently outperformed those on CIC-IDS2017, indicating that the proposed approach is particularly effective for detecting DDoS attacks. In contrast, more complex intrusion patterns found in CIC-IDS2017 may require additional sources of information, which are beyond network traffic features alone.

Despite these promising results, several limitations must be acknowledged. First, the experiments were conducted on simulated non-IID distributions derived from the CIC-IDS2017 and CIC-IDS2019 datasets. Although different partitioning strategies were employed, with Dirichlet partitioning being a well-established approach, these methods may not fully capture the complexity of real-world traffic heterogeneity across distributed networks. Second, performance metrics such as F1-score and accuracy were evaluated on balanced test sets to enable fair comparisons across models and accommodate hardware constraints. While this ensured consistency, it does not reflect real-world environments where class distributions are often highly imbalanced. Third, the thresholding process for anomaly detection was tuned using labeled data to maximize F1-scores. Although effective in experimental settings, this strategy is less feasible in practice, where labeled data is scarce and threshold selection involves operational trade-offs. Fourth, the federated learning experiments were conducted in a controlled simulation environment with limited computational resources. Due to the large number of experiments required, per-client performance assessments were not the focus, which limits the granularity of insights into client-level variability. Finally, critical system-level factors, such as communication overhead, network latency, client dropouts, and large-scale heterogeneity, were not captured in this study.

Building on this study, several promising directions for future research can be pursued. First, more advanced forms of autoencoders, such as Variational Autoencoders (VAEs) and Adversarial Autoencoders (AAEs), could be explored to enhance anomaly detection performance. Second, deploying real-world federated testbeds would allow researchers to account for practical challenges, including communication costs, client availability, and system heterogeneity, which are absent in simulated environments but critical in practice. Third, extending the approach to broader domains beyond DDoS detection, such as intrusion detection or malware classification, would provide stronger evidence of the framework’s generalizability. Fourth, threshold selection strategies should be further investigated, including personalized tuning for each client and adaptive thresholding mechanisms that adjust dynamically based on system load. Finally, evaluating additional federated optimization strategies, such as FedDyn (a dynamic federated regularization technique) or Ditto (a personalized federated learning method), could further improve adaptability and robustness across diverse network environments.

In summary, this research has demonstrated that federated learning, when combined with autoencoder-based anomaly detection, provides a viable and effective framework for addressing the challenges of distributed DDoS detection in heterogeneous environments. By evaluating multiple optimization strategies and testing across benchmark datasets, the study offered empirical evidence of the strengths and weaknesses of current approaches while highlighting the suitability of network-traffic–based metrics for DDoS scenarios. While certain limitations remain, the results confirm the potential of this line of research to advance resilient, decentralized cybersecurity solutions. With continued refinement of models, incorporation of real-world constraints, and extension to broader security domains, federated learning holds promise as a foundational technology in safeguarding future networks against large-scale threats.
